\documentclass{article}
%\usepackage[margin=1.2in]{geometry}

\usepackage{nips15submit_e}


% For figures
\usepackage{graphicx}
\usepackage{subfigure}
\usepackage{color}

% For citations
\usepackage{natbib}

% For algorithms
\usepackage{algorithm}
\usepackage{algorithmic}

% For links
\usepackage{hyperref}
\newcommand{\theHalgorithm}{\arabic{algorithm}}

\newcommand{\glnote}[1]{\textcolor{red}{GL: #1}}


% Header ======================================================================

\title{Semi-supervised author disambiguation of scientific publications}

\author{Gilles Louppe\\
        CERN\\
        Switzerland\\
\And Hussein Al-Natsheh\\
        CERN\\
        Switzerland\\
\And Mateusz Susik\\
        CERN\\
        Switzerland\\
\And Eamonn Maguire\\
        CERN\\
        Switzerland}
\date{}

\nipsfinalcopy

\begin{document}

\maketitle

\begin{abstract}

Author name disambiguation in bibliographic databases is the problem of
grouping together scientific publications written by a same person, accounting
for potential homonyms and/or synonyms. Among solutions to this problem, digital
libraries are increasingly offering tools for authors to manually curate their
publications and claim which ones are theirs. Indirectly, these tools allow for
the inexpensive collection of large annotated training data, which can be
further leveraged to build an automated disambiguation system capable of
inferring features or patterns for identifying publications written by a same
person.  In this context, we propose a semi-supervised machine learning
disambiguation solution in the case large and partially annotated databases of
scientific publications. Using a real-world dataset of 1 million manual
annotations, we demonstrate how we exploited this data (i) to learn an accurate
classifier for identifying corefering authors, (ii) to guide the clustering of
scientific publications by distinct authors in a semi-supervised way and
finally (iii) to match the resulting clusters with existing profiles.  Our
analysis of parameters and features on this large training set reveal that...
\glnote{To be completed with actual results.}

\end{abstract}


% Introduction ==================================================================

\section{Introduction}
\label{introduction}

% general introduction to the disambiguation issues
% often based on small dataset or fully unsupervised
% summary of what we did
% outline of the paper

In academic digital libraries, author name disambiguation is the problem of
grouping together publications written by a same person.  With the fast growth
of the scientific literature, this has become a pressing issue because the
accuracy of information managed at the level of individuals directly affects
the relevance search results (e.g., when querying for all publications written
by a given author), the reliability of bibliometrics (e.g., citation counts or
other impact metrics) or the relevance of scientific network analysis. Author
name disambiguation is often a difficult problem because an author may use
different spellings, variants or synonyms across her career and/or distinct
authors may share the same name. In particular, author disambiguation becomes
even more critical for researchers from non-Western cultures, where personal names
may be traditionally less diverse (leading to homonym issues) or for which
transliteration to latin characters may not be unique (leading to synonym
issues).

Efforts and solutions to author disambiguation have been proposed from various
communities \citep{liu2014author}. On the one hand, libraries have maintained
authorship control through manual curation, either in a centralized way by
hiring collaborators or by developing services inviting authors to register
their publications themselves (e.g., Google Scholar, Inspire-HEP, ...). Recently,
efforts also started to create persistent digital identifiers assigned to researchers
(e.g., ORCID, ResearchedID, ...), with the further objective to embed these
identifiers in the submission workflow of publishers or repositeries (e.g.,
Elsevier, arXiv, Inspire-HEP, ...), thereby univocally solving any disambiguation
issue. With the large cost of centralized manual authorship control or until
the larger adoption of crowdsourced solutions, the impact of these efforts are
unfortunately limited by the efficiency, motivation or integrity of their
active contributors. Similarly, the success of persistent digital identifier
efforts is conditioned to a large and ubiquituous adoption by both researchers and publishers.
For these reasons, fully automated machine learning-based methods have been
proposed during the past decade \glnote{Add references} to provide immediate, less costly
and efficient solutions to author disambiguation. In this work, our goal
is to explore and demonstrate how both approaches can coexist and benefit from
each other.  In particular, we study how labeled data obtained through manual
curation (either centralized or crowdsourced) can be exploited (i) to learn an
accurate classifier for identifying corefering authors, (ii) to guide the
clustering of scientific publications by distinct authors in a semi-supervised
way and finally (iii) to match the resulting clusters with existing profiles.
Our analysis of parameters and features on this large training set reveal
that... \glnote{To be completed with actual results.}

The remaining of this report is structured as follows. In Section~\ref{related-works}, ...


% Related works ==================================================================

\section{Related works}
\label{related-works}

\glnote{I will take care of this}

% levin

% Methods =====================================================================

\section{Semi-supervised author disambiguation}
\label{methods}

\subsection{Overview}
% figure, notations, etc

Formally, let us assume a set of publications ${\cal P} = \{ p_0, ...,
p_{N-1}\}$ and a set of authors ${\cal A} = \{ a_0, ..., a_{M-1}\}$
corresponding to individuals.  Let us define a signature $s \in p$ from a
publication as a unique piece of information identifying one of the authors of
$p$ (e.g., the author name, his affiliation, along with any other metadata can
be derived from $p$). Let us denote by ${\cal S} = \{ s | s \in p, p \in {\cal
P} \}$ the set of all signatures that can be extracted from all publications in
${\cal P}$. In this context, the author disambiguation problem can be stated as
the problem of finding a partition ${\cal C} = \{ c_0, ..., c_{M-1} \}$ of
${\cal S}$ such that ${\cal S} = \cup_{i=0}^{M-1} c_i$, $c_i \cap c_j = \phi$
for all $i \neq j$, and where subsets $c_i$, or clusters, each corresponds to
the set of all signatures belonging to the same individual $a_i$. Note that in
practice, the set ${\cal A}$ often remains unknown and the task therefore boils
down to finding a partition ${\cal C}$ such that subsets   $c_i$  each
corresponds to the set of all signatures from a same individual.



% Before presenting the methodology, we need to define some terminologies we will
% use to describe the data entities. An author's full name, or author name, is
% the string variation that was mention in the database record to refer to the
% author. The full name is a composition of the author last name and the author
% other names. For example, in the full name (Kerth, Leroy T.), the last name is
% Kerth and the other names will be (Leroy T.). We also use the terminology
% initials. In our full name example, the initials will be (L T). First initial
% then will be (L). So, grouping publications  What we refer to as the real
% author is the real person behind the publication. We will use the term
% signature as the entity that we apply clustering on.  A signature is a unique
% combination between a publication and an author name as mentioned in the
% database. The term blocking means grouping the signature by a certain version
% of the author name. For example, blocking by the author last name and first
% name initial means grouping together all signatures that have in common the
% same author's last name and first name initial. In our example, grouping by the
% string (Kerth L). What we mean by claimed publications is the publications in
% which their real author identity were claimed. We use these claimed
% publications to generate our ground truth data.

% Our methodology consists in 4 main components: Distance learning, blocking,
% semi-supervised clustering and matching. A distance function is learned from a
% training set of signiture pairs sampled from ground-truth data. Each pair is
% labeled whether the 2 signatures are for the same real author or not. The
% learned distance function will then be used for computing the dissimilarity
% matrix of a block of signatures. A block of signatures is grouping signatures
% that have in common similar author names, for example same last name. This
% affinity matrix would be the main input of our semi-supervised clustering
% algorithm. The predicted clusters of each block will lastly be combined into a
% final predicted clusters result. Each cluster label represents the predicted
% single author of the publications in the cluster. Finally, we present a
% strategy for matching the predicted results with the publications in which
% their authors were claimed.

\subsection{Distance learning}
% distance learning as a supervised problem
% balancing difficult and easy cases

\subsection{Blocking}
% motivation
% last name, first initial
% phonetics something (dm)

\subsection{Semi-supervised clustering}
% hierarchical clustering
% semi-supervised clustering
% cold start problem (when no labels in a block)

\subsection{Matching}
% general description in a real system


% Implementation ======================================================

\section{Implementation}
\label{implementation}


% Results and discussion ======================================================

\section{Experiments}
\label{experiments}

\subsection{Data}
% source of the data, labels in the data, etc
% release the data

\subsection{Evaluation protocol}
% measures (b^3, paired, size of clusters)
% training / valid / test sets

\subsection{Distance learning}
% feature engineering
% variable importance analysis
% supervised learning model

\subsection{Blocking}

\subsection{Semi-supervised clustering}
% compare agglomerative algorithms
% compare threshold strategies
% compare cold start solutions
% talk about impact of small improvements in terms of citations statistics


% Conclusions =================================================================

\section{Conclusions}
\label{conclusions}

% future improvements


% References ==================================================================

\bibliographystyle{apalike}
\bibliography{bib}

\end{document}
